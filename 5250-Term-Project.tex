% Options for packages loaded elsewhere
\PassOptionsToPackage{unicode}{hyperref}
\PassOptionsToPackage{hyphens}{url}
%
\documentclass[
]{article}
\title{5250 Term project}
\author{Satish Mahabhashyam (0721750)}
\date{}

\usepackage{amsmath,amssymb}
\usepackage{lmodern}
\usepackage{iftex}
\ifPDFTeX
  \usepackage[T1]{fontenc}
  \usepackage[utf8]{inputenc}
  \usepackage{textcomp} % provide euro and other symbols
\else % if luatex or xetex
  \usepackage{unicode-math}
  \defaultfontfeatures{Scale=MatchLowercase}
  \defaultfontfeatures[\rmfamily]{Ligatures=TeX,Scale=1}
\fi
% Use upquote if available, for straight quotes in verbatim environments
\IfFileExists{upquote.sty}{\usepackage{upquote}}{}
\IfFileExists{microtype.sty}{% use microtype if available
  \usepackage[]{microtype}
  \UseMicrotypeSet[protrusion]{basicmath} % disable protrusion for tt fonts
}{}
\makeatletter
\@ifundefined{KOMAClassName}{% if non-KOMA class
  \IfFileExists{parskip.sty}{%
    \usepackage{parskip}
  }{% else
    \setlength{\parindent}{0pt}
    \setlength{\parskip}{6pt plus 2pt minus 1pt}}
}{% if KOMA class
  \KOMAoptions{parskip=half}}
\makeatother
\usepackage{xcolor}
\IfFileExists{xurl.sty}{\usepackage{xurl}}{} % add URL line breaks if available
\IfFileExists{bookmark.sty}{\usepackage{bookmark}}{\usepackage{hyperref}}
\hypersetup{
  pdftitle={5250 Term project},
  pdfauthor={Satish Mahabhashyam (0721750)},
  hidelinks,
  pdfcreator={LaTeX via pandoc}}
\urlstyle{same} % disable monospaced font for URLs
\usepackage[margin=1in]{geometry}
\usepackage{graphicx}
\makeatletter
\def\maxwidth{\ifdim\Gin@nat@width>\linewidth\linewidth\else\Gin@nat@width\fi}
\def\maxheight{\ifdim\Gin@nat@height>\textheight\textheight\else\Gin@nat@height\fi}
\makeatother
% Scale images if necessary, so that they will not overflow the page
% margins by default, and it is still possible to overwrite the defaults
% using explicit options in \includegraphics[width, height, ...]{}
\setkeys{Gin}{width=\maxwidth,height=\maxheight,keepaspectratio}
% Set default figure placement to htbp
\makeatletter
\def\fps@figure{htbp}
\makeatother
\setlength{\emergencystretch}{3em} % prevent overfull lines
\providecommand{\tightlist}{%
  \setlength{\itemsep}{0pt}\setlength{\parskip}{0pt}}
\setcounter{secnumdepth}{-\maxdimen} % remove section numbering
\ifLuaTeX
  \usepackage{selnolig}  % disable illegal ligatures
\fi

\begin{document}
\maketitle

\(if(geometry)\)

\usepackage[$for(geometry)$$geometry$$sep$,$endfor$]{geometry}

\(endif\)

\hypertarget{facebook-ad-analysis}{%
\section{Facebook Ad Analysis}\label{facebook-ad-analysis}}

\hypertarget{literature-review}{%
\subsection{Literature Review}\label{literature-review}}

The Internet has changed the world in ways that were unimaginable a
couple of decades back. This method of connectivity on such a massive
scale has given businesses and individuals uncountable opportunities.
With the growing population and increasing numbers of people in the
digital world, we are seeing a constant rise in data. This ginormous
amount of data is termed Big Data. Big Data is a seemingly complex term
but has a simple explanation. To put it simply, we can take massive
amounts of data and derive a story from it. This was needed because a
few years ago all this data was mostly gibberish to us. With the
advancements in distributed computing and parallel processing, this
equation has completely changed. We can now analyze this data, detect
patterns \& derive predictions from it. Big Data is being used by
businesses to predict the customer's needs, pain points \& future
complaints. These implications are huge and can increase a company's
profits tremendously. It is shocking to see how efficiently the new
digital marketing strategies work. Smaller companies, with small
budgets, are also capitalizing on Big Data. Research carried out by EMC
Forum 2013 indicates that 39\% of entrepreneurs believe that Big Data
provides business success {[}1{]}.

Big data consists of various types of data. It can contain text,
documents, multimedia files, transactions, financial records, web server
logs and streaming data from sensors. These varieties of data are
majorly categorized into three categories -- structured, semi-structured
and unstructured. These categories have different roles in helping
create a data model. More about these can be found in the paper
``Influence of structured, semi-structured, unstructured data on various
data models'' {[}2{]}. Big data management is mostly characterized by 3
Vs -- Volume, Velocity and Variety. These 3Vs play a significant role in
distinguishing big data from traditional data. Previous research has
shown how social media sites like Facebook and Twitter account for a
large percentage of digital content {[}3{]}.

Advertisements have massively changed over the last 50 years. Previous
researches show that with the advancement in technology consumers are
now in control of the media message they want to become exposed to
{[}4{]}. Targeted Facebook ads are an especially good example of this.
When it comes to `Personal' information, Facebook's data collection is
unmatched. Facebook can predict what users will be interested in
purchasing in the coming days, weeks \& months. It does it by collecting
massive amounts of even the most MINUTE data, like how long your cursor
hovers over a certain link. In their paper, Jan-Willem van Dam and
Michel van de Velden explain show how data from Facebook is used to gain
insight into the individuals having an account on the Facebook site
{[}5{]}. With their most recent acquisition of Oculus {[}6{]}, Facebook
can now track \& record, how long a user looks at a certain object. This
strategy of acquiring data by Facebook allows its clients to
cost-effectively target their audience with shocking accuracy. Big Data
is changing the landscape of technology and business.

Using the personal data of users, Facebook makes a profile of users
based on their likes, dislikes, and their behavior. They do so to create
sophisticated profiles of the user for predictive purposes. The article
shows how companies like Facebook and Google/Alphabet and a few
third-party data brokers collect and combine detailed personal data to
create sophisticated profiles {[}15{]}. This is the reason why Facebook
seems attractive to advertisers. More about why it is so can be read in
the paper ``The effectiveness of Facebook Advertising by the Degree of
its benefits to Advertisers'' by Ahmet Ertugan {[}7{]}. This gives the
advertisers the ability to not only build their own adverts but also,
design their own audience for the advert they created. Audiences can be
built from a range of attributes including age, gender, the user's
location, and interests. Even though the product being marketed is the
same, with such precise methods of targeting users, advertisers can
tailor content perfectly for a specific audience. This gives them a
massive edge over other businesses.

In 2016, Facebook's revenue from advertising was \$26bn, up from \$17bn
the year before {[}8{]}. This compares to Google's \$79bn {[}9{]} the
\$638m that Twitter advertising made in Q4 2016 {[}10{]} and \$173m that
LinkedIn made from ads in Q3 2016 {[}11{]}. Even though Facebook lost
many young users back in 2017, it still holds second place after Google
in terms of ad revenue today.

\end{document}
